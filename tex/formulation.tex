\documentclass[]{article}

\usepackage[a4paper,twoside,margin=3.0cm]{geometry}

\usepackage{amsmath}
\usepackage{stmaryrd}
\usepackage{cases}
\usepackage{tikz}
\usepackage{tikz-cd}

\newcommand*{\llbraces}{\{\mskip-5mu\{}
\newcommand*{\rrbraces}{\}\mskip-5mu\}}

\newcommand{\Gunnar}[1]{\textcolor{violet}{G: }\textcolor{violet}{#1}}  

%opening
\title{}
\author{}

\begin{document}

\maketitle

\section{Analytic equation}

We start with the the micro-macro decomposition considered in \cite{JaLiQiXi2014} which is given by
\begin{subequations}
\begin{align}
& \partial_t \rho + \partial_x \langle v g \rangle = 0 \label{eq: mmeq1} \\
& \partial_t g + \frac{1}{\epsilon} (\mathbf{I} - \Pi) (v \partial_x g) + \frac{1}{\epsilon^2} v \partial_x \rho= \frac{1}{\epsilon^3} \mathcal{C}(\rho + \epsilon g) \label{eq: mmeq2}
\end{align}
\end{subequations}

Writing $g = (\alpha, \beta)$ we can expand the second line \eqref{eq: mmeq2} into
\begin{align*}
\partial_t \alpha + \frac{1}{2 \epsilon} \partial_x (\alpha + \beta)+ \frac{1}{\epsilon^2} \partial_x \rho = & \frac{1}{\epsilon^3} \mathcal{C}(\rho + \epsilon \alpha)\\
\partial_t \beta - \frac{1}{2\epsilon} \partial_x (\alpha + \beta)- \frac{1}{\epsilon^2} \partial_x \rho = & \frac{1}{\epsilon^3} \mathcal{C}(\rho + \epsilon \beta)
\end{align*}
Here we used that
\[
\Pi (v \partial_x g) = \frac{1}{2}(\alpha - \beta)
\]
Setting $j = \frac{1}{2}(\alpha - \beta)$ and subtracting the second line from the first we get
\[
2 \partial_t j + \frac{1}{\epsilon} \partial_x (\alpha + \beta) + \frac{2}{\epsilon^2} \partial_x \rho = \frac{1}{\epsilon^3} (\mathcal{C}(\rho + \epsilon \alpha) - \mathcal{C}(\rho + \epsilon \beta)) 
\]
Using $\alpha = -\beta$ due to $\langle g \rangle = 0$ and dividing by $2$ we recover the telegraph equation
\begin{align*}
& \partial_t \rho + \partial_x j = 0\\
& \partial_t j + \frac{1}{\epsilon^2} \partial_x \rho = \frac{1}{2\epsilon^3} (\mathcal{C}(\rho + \epsilon \alpha) - \mathcal{C}(\rho + \epsilon \beta)) 
\end{align*}

\section{Numerical scheme}

\subsection{Space discretization}

\subsubsection{Cut cell grid}

Our spatial domain discretizations starts off with $N+1$ equidistant grid points $(x_{i - \frac{1}{2}})_{(1 \le i \le N + 1)}$ on the domain $(a, b)$ forming a grid with $N$ elements  $E_i = (x_{i - \frac{1}{2}}, x_{i + \frac{1}{2}})$ of size $|E_i| = h_i = x_{i + \frac{i}{2}} - x_{i - \frac{1}{2}}$.

Let for some $1 \le k \le N$ a point $x_{cut} \in E_k$  be given, seperating $E_k$ into two cut cells $E_{cut, 1} = (x_{k - \frac{1}{2}}, x_{cut})$ and $E_{cut, 2} = (x_{cut}, x_{k + \frac{1}{2}})$.  This yields the cut cell grid
\[
\mathcal{M}_h = \{ E_i : i \neq k\} \cup \{ E_{cut, 1}, E_{cut, 2} \}.
\]
We will assume that $E_{cut}^1$ is the small cell with $|E_i| \ll h$, requiring additional handling if a timestep size depending on $h$ is chosen.

We define our discrete function space as
\[
V_h = \{ v_h \in L^2((a, b)) : (v_h)_{|E} \in \mathcal{P}^k(E), \; E \in \mathcal{M}_h \}
\]
that means our discrete functions are piecewise polynomials on the cut cell grid.

\subsubsection{Base scheme}

The semi-discretization in space can be written as

\begin{subequations}
\begin{align}
& (\partial_t \rho_h, \phi_h) + a_h(g_h, \phi_h) = 0\\
& (\partial_t g_h, \psi_h) + \frac{1}{\epsilon}b_h(g_h, \psi_h) + \frac{v}{\epsilon^2}c_h(\rho_h, \psi_h) = s_h(\rho_h, g_h, \psi_h, v)
\end{align}
\end{subequations}

with bilinear forms
\begin{subequations}
\begin{align}
	\begin{split}
a_h(g_h, \phi_h) = & - \sum_{E \in \mathcal{M}_h} \int_{E} \langle v g_h \rangle \partial_x \phi_h \\ & - \sum_{i} h_2(\langle v g_h \rangle^-, \langle v g_h \rangle^+) \llbracket \phi_h \rrbracket_{x_{i - \frac{1}{2}}} -h_2(\langle v g_h \rangle^-, \langle v g_h \rangle^+) \llbracket \phi_h \rrbracket_{x_{cut}}
\end{split}\\
\begin{split}
b_h(g_h, \psi_h, v) = & - \sum_{E \in \mathcal{M}_h} \int_{E} v g_h \partial_x \psi_h \\ & - \sum_i h_1(g_h^-, g_h^+, v)  \llbracket \psi_h \rrbracket_{x_{i - \frac{1}{2}}} - h_1(g_h^-, g_h^+, v) \llbracket \psi_h \rrbracket_{x_{cut}}
\end{split}\\
\begin{split}
c_h(\rho_h, \psi_h) = & - \sum_{E \in \mathcal{M}_h}\int_{E} \rho_h \partial_x \psi_h \\ & - \sum_i h_2(\rho_h^-, \rho_h^+) \llbracket \psi_h \rrbracket_{x_{i - \frac{1}{2}}} - h_2(\rho_h^-, \rho_h^+) \llbracket \psi_h \rrbracket_{x_{cut}}
\end{split}\\
s_h(\rho_h, g_h, \psi_h, v) = & - \frac{1}{\epsilon^2}s_h^1(\rho_h, g_h, \psi_h, v) - s_h^2(g_h, \psi_h, v)
\end{align}
\end{subequations}

where the definitions of $s_h^1(\rho_h, g_h, \psi_h, v)$ and $s_h^2(g_h, \psi_h, v)$ depend on the choice of the collision operator:

	\begin{align*}
		s_h^1(\rho_h, g_h, \psi_h, v) = \begin{cases}
			(K \rho_h^m g_h, \psi) \\
			(g_h - A \nu \rho_h, \psi_h) \\
			(g_h - C \nu \rho_h^2, \psi_h)
		\end{cases}
		\quad
		s_h^2(g_h, \psi_h, v) = \begin{cases}
			0 \\
			0 \\
			(C v g_h^2, \psi_h)
		\end{cases}
	\end{align*}
	
	The numerical fluxes are chosen as
	\begin{subequations}
\begin{align}
h_1(u_l, u_r, v) & =  \frac{v}{2} (u_l + u_r)  - \frac{|v|}{2} (u_r - u_l) =\begin{cases}
v u_r,  \quad v < 0 \\
v u_l, \quad v > 0
\end{cases} \\
h_2(u_l, u_r) & = \frac{1}{2} (u_l + u_r)
\end{align}
	\end{subequations}
	\Gunnar{Using a central flux for $h_2$ should be fine for the moment, but we might need to change that later.}
	
	\subsubsection{DoD scheme}
	
For our DoD scheme we define two more bilinear forms
\begin{subequations}
\begin{align}
	\begin{split}
\frac{1}{\eta} a_h^{DoD}(g_h, \phi_h) = & - \int_{E_{cut, 1}} \sum_{j} \left(\langle v g_h \rangle - h_2 (\langle v g_h \rangle^-, \langle v g_h \rangle^+)  \right) \partial_x \phi_h \\ 
& - \int_{E_{cut, 1}} \sum_{j} \left(\langle v g_h \rangle - h_2 (\langle v g_h \rangle^-, \langle v g_h \rangle^+)  \right) \partial_x \phi_h \\
& - \int_{E_{cut, 1}} \sum_{j} \left(\langle v g_h \rangle - h_2 (\langle v g_h \rangle^-, \langle v g_h \rangle^+)  \right) \partial_x \phi_h \\ & -  h_2 (\langle v g_h \rangle^-, \langle v g_h \rangle^+)  \llbracket \phi_h \rrbracket_{x_{k-\frac{1}{2}}} -  h_2 (\langle v g_h \rangle^-, \langle v g_h \rangle^+)  \llbracket \phi_h \rrbracket_{x_{cut}}
\end{split}\\
\begin{split}
\frac{1}{\eta}  b_h^{DoD}(g_h, \psi_h, v) = & - \int_{E_{cut, 1}} \big( v g_h - h_1(g_h^-, g_h^+, v)  \big) \partial_x \psi_h \\ & -  h_1(g_h^-, g_h^+, v)  \llbracket \psi_h \rrbracket_{x_{k-\frac{1}{2}}}  -  h_1(g_h^-, g_h^+, v) \llbracket \psi_h \rrbracket_{x_{cut}}
\end{split}
\end{align}
\end{subequations}
This then gives the scheme
\begin{subequations}
	\begin{align}
		& (\partial_t \rho_h, \phi_h) + \tilde{a}_h(g_h, \phi_h) = 0\\
		& (\partial_t g_h, \psi_h) + \frac{1}{\epsilon}\tilde{b}_h(g_h, \psi_h) + \frac{v}{\epsilon^2}c_h(\rho_h, \psi_h) = s_h(\rho_h, g_h, \psi_h, v)
	\end{align}
\end{subequations}

with
\begin{subequations}
	\begin{align}
		\tilde{a}_h(g_h, \phi_h) & = a_h(g_h, \phi_h)  + a_h^{DoD}(g_h, \phi_h) \\
		\tilde{b}_h(g_h, \psi_h, v) & = b_h(g_h, \psi_h, v)  + b_h^{DoD}(g_h, \psi_h, v) 
	\end{align}
\end{subequations}

\subsection{Time discretization}

A first order IMEX discretization is given by

\begin{subequations}
\begin{align}
\label{eq: first order imex scheme 1}  \frac{1}{\Delta t} (\rho_h^{n+1} - \rho_h^n, \phi_h) + a_h(g_h, \phi_h) & = 0 \\
\label{eq: first order imex scheme 2} \frac{1}{\Delta t} (g_h^{n+1} - g_h^n, \psi_h) + \frac{1}{\epsilon} b_h(g_h^n, \psi_h, v) + \frac{v}{\epsilon^2} c_h(\rho_h^{n+1}, \psi_h) & = - \frac{1}{\epsilon^2}s_h^1(\rho_h^{n+1}, g_h^{n+1}, \psi_h, v) - s_h^2(g_h^n, \psi_h, v)
\end{align}
\end{subequations}
From this we can already read off that $a_h$ in \eqref{eq: first order imex scheme 1} will probably need a DoD stabilization since the equation is handled purely explicitly. We might also expect that $b_h$ in \eqref{eq: first order imex scheme 2} will need a DoD stabilization to prevent a blowup of the RHS. On the other hand $c_h$ in \eqref{eq: first order imex scheme 2} should require no DoD stabilization since it is handled implicitly.

We further note that $b_h$ will have no impact on asymptotic stability, at least formally, since it vanishes in the (formal) limit. This should perhaps be checked rigorously, though.

\section{Stability analysis}

\subsection{Asymptotic stability}

We start by taking the formal limit $\epsilon \to 0$ in \eqref{eq: first order imex scheme 2}. This gives
\[
v c_h (\rho_h^{n+1}, \psi_h) = - s_h^1 (\rho_h^{n+1}, g_h^{n+1}, \psi_h, v)
\]
For simplicity we only cosider the case
\[
s_h^1(\rho_h, g_h, \psi_h, v) = (g_h, \psi_h), \quad s_h^2(g_h, \psi_h, v) = 0
\]
Multiplying \eqref{eq: first order imex scheme 2} by $v$ and applying the $\langle \cdot \rangle$ operation we get
\[
c_h(\rho_h^{n+1}, \psi_h) = (\langle vg_h \rangle, \psi_h)
\]
Together with \eqref{eq: first order imex scheme 1} we arrive at the scheme
\begin{align*}
	\frac{1}{\Delta t} (\rho_h^{n+1} - \rho_h^n, \phi_h) - \sum_i \int_{E_i} \langle v g_h \rangle \partial_x \phi_h - \sum_{i} \widehat{\langle v g \rangle} \llbracket \phi_h \rrbracket_{i-\frac{1}{2}} & = 0\\
	- \sum_i \int_{E_i} \rho_h \partial_x \psi_h - \sum_i \widehat{ \rho_h} \llbracket \psi_h \rrbracket_{i - \frac{1}{2}}  & = (\langle vg_h \rangle, \psi_h)
\end{align*}

This should be a scheme for the heat equation with a mixed formulation for the laplacian, i. e.
\begin{align*}
\partial_t \rho + \partial_x \sigma & = 0 \\
\sigma & = \partial_x \rho
\end{align*}
with $\sigma := \langle v \rho \rangle$.

\subsection{Julia Implementation}
We want to implement the telegraph equation. 
Therefore, we have $b_h=0$ and with $K=1, m=0$ $s_h=-\frac{1}{\epsilon^2}(g_h,\psi_h)$.\\
%Remind also, that with $g=(\alpha, \beta)$ and $\langle g\rangle=0$, we have $\alpha = -\beta$ and therefore
%$$j(x,t)=\frac{1}{2\epsilon}\left(f(x,v=1,t)-f(x,v=-1,t)\right)=\frac{1}{2\epsilon}\left(\rho+\epsilon\alpha-\rho-\epsilon\beta\right)=\frac{1}{\epsilon}\alpha$$.
So we end up with the semidiscretization 
\begin{align*}
(\partial_t\rho_h, \phi_h)&=\sum\limits_i \int_{E_i} \langle j_h \rangle \partial_x \phi_h + \sum_{i} \widehat{\langle j \rangle} \llbracket \phi_h \rrbracket_{i-\frac{1}{2}}\\
(\partial_t j_h, \psi_h)&= \frac{1}{\epsilon^2}\left(\sum_i \int_{E_i} \rho_h \partial_x \psi_h + \sum_i \widehat{ \rho_h} \llbracket \psi_h \rrbracket_{i - \frac{1}{2}} + \left(j_h, \psi_h\right)\right),
\end{align*}
which translates into the matrix-formulation
\begin{equation*}
    \begin{pmatrix}
        SM \rho' \\
        SM j'
    \end{pmatrix} =
    \begin{pmatrix}
        -Bj^*+D^TMj \\
        \frac{1}{\epsilon^2}\left(-B\rho^*+D^TM\rho + Mj\right)
    \end{pmatrix}
\end{equation*}
with SBP operators $D, M$ and flux operator $B$ (or $V^TBV$). Further, $S$ is the scaling matrix $S=diag(\Delta x_1/2, \Delta x_2/2, \hdots, \Delta x_N/2)$ and $\rho^*, j^*$ are the boundary values.\\
To treat the flux terms
\begin{equation*}
\begin{pmatrix}
    Bj^* \\
    B\rho^*
\end{pmatrix}
\end{equation*}
one coud consider the choices of the paper \cite{JaLiQiXi2014}, which alternate the directions. But what is about the splitting

\begin{equation*}
A^*\left(\begin{pmatrix} \rho_{i-1}\\j_{i-1} \end{pmatrix} - \begin{pmatrix} \rho_i\\ j_i\end{pmatrix} \right) \begin{pmatrix} \phi_h^-\\ \psi_h^-\end{pmatrix} 
+ A^- \left(\begin{pmatrix} \rho_{i-1}\\j_{i-1} \end{pmatrix} - \begin{pmatrix} \rho_i\\ j_i \end{pmatrix} \right) \begin{pmatrix} \phi_h^+\\ \psi_h^+\end{pmatrix} ?
\end{equation*}


\begin{thebibliography}{9}
	\bibitem{JaLiQiXi2014}
	Juhi Jang, Fengyan Li, Jing-Mei Qiu, Tao Xiong,
	High order asymptotic preserving DG-IMEX schemes for discrete-velocity kinetic equations in a diffusive scaling,
	Journal of Computational Physics,
	Volume 281,
	2015,
	Pages 199-224,
	ISSN 0021-9991,
	https://doi.org/10.1016/j.jcp.2014.10.025.
	(https://www.sciencedirect.com/science/article/pii/S0021999114007074)
	
	\bibitem{JaLiQiXi20142}
		Jang, Juhi and Li, Fengyan and Qiu, Jing-Mei and Xiong, Tao,
		Analysis of Asymptotic Preserving DG-IMEX Schemes for Linear Kinetic Transport Equations in a Diffusive Scaling
		SIAM Journal on Numerical Analysis
		Volume 52, 4
		pages 2048-2072,
		2014,
		doi 10.1137/130938955,
			https://doi.org/10.1137/130938955


\end{thebibliography}

\end{document}
