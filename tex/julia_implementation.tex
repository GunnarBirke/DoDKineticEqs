\subsection{Julia Implementation}
We want to implement the telegraph equation. 
Therefore, we have $b_h=0$ and with $K=1, m=0$ $s_h=-\frac{1}{\epsilon^2}(g_h,\psi_h)$.\\
%Remind also, that with $g=(\alpha, \beta)$ and $\langle g\rangle=0$, we have $\alpha = -\beta$ and therefore
%$$j(x,t)=\frac{1}{2\epsilon}\left(f(x,v=1,t)-f(x,v=-1,t)\right)=\frac{1}{2\epsilon}\left(\rho+\epsilon\alpha-\rho-\epsilon\beta\right)=\frac{1}{\epsilon}\alpha$$.
So we end up with the semidiscretization 
\begin{align*}
(\partial_t\rho_h, \phi_h)&=\sum\limits_i \int_{E_i} \langle j_h \rangle \partial_x \phi_h + \sum_{i} \widehat{\langle j \rangle} \llbracket \phi_h \rrbracket_{i-\frac{1}{2}}\\
(\partial_t j_h, \psi_h)&= \frac{1}{\epsilon^2}\left(\sum_i \int_{E_i} \rho_h \partial_x \psi_h + \sum_i \widehat{ \rho_h} \llbracket \psi_h \rrbracket_{i - \frac{1}{2}} + \left(j_h, \psi_h\right)\right),
\end{align*}
which translates into the matrix-formulation
\begin{equation*}
    \begin{pmatrix}
        SM \rho' \\
        SM j'
    \end{pmatrix} =
    \begin{pmatrix}
        -Bj^*+D^TMj \\
        \frac{1}{\epsilon^2}\left(-B\rho^*+D^TM\rho + Mj\right)
    \end{pmatrix}
\end{equation*}
with SBP operators $D, M$ and flux operator $B$ (or $V^TBV$). Further, $S$ is the scaling matrix $S=diag(\Delta x_1/2, \Delta x_2/2, \hdots, \Delta x_N/2)$ and $\rho^*, j^*$ are the boundary values.\\
To treat the flux terms
\begin{equation*}
\begin{pmatrix}
    Bj^* \\
    B\rho^*
\end{pmatrix}
\end{equation*}
one coud consider the choices of the paper \cite{JaLiQiXi2014}, which alternate the directions. But what is about the splitting

\begin{equation*}
A^*\left(\begin{pmatrix} \rho_{i-1}\\j_{i-1} \end{pmatrix} - \begin{pmatrix} \rho_i\\ j_i\end{pmatrix} \right) \begin{pmatrix} \phi_h^-\\ \psi_h^-\end{pmatrix} 
+ A^- \left(\begin{pmatrix} \rho_{i-1}\\j_{i-1} \end{pmatrix} - \begin{pmatrix} \rho_i\\ j_i \end{pmatrix} \right) \begin{pmatrix} \phi_h^+\\ \psi_h^+\end{pmatrix} ?
\end{equation*}
